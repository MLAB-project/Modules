\documentclass[12pt,a4paper,final,titlepage,twoside]{article}
\usepackage{MLABdoc}
\fancyhead[L]{{\huge SHT31V01A}}
\fancyfoot[RE,LO]{\today |  Autor1,  Autor2 | mlab.cz}

\includeonly{SHT31V01A.cs.title, SHT31V01A.cs.components}

\begin{document}
%vytvori uvodni stranku
\uvod
% nazev modulu
{SHT31V01A}
% kratky popis
{digitální vlhkoměr a teploměr}
%autor/i
{ Autor 1,  Autor 2}
%obrazek
{../img/SHT31V01A_top_big.jpg}
%abstrakt
{Jedná se o modul, který je možné osadit IO SHT30 nebo SHT31, které umí měřit relativní vlhkost a teplotu s velkou přesností a stabilitou. Rozsah měřené vlhkosti je 0\%  až 100\% . Teplota je měřena v rozsahu -40  $^\circ$C až 125 $^\circ$C. Komunikace probíhá přes rozhranní I2C.}
%tabulka
{  Relativní vlhkost & 0\% - 100\% & Typ. přesnost dle IO\\ \hline   Rozhraní & I2C & \\ \hline   Teplota & 0 $^\circ$ C - 100 $^\circ$C & Typ. přesnost dle IO\\ \hline   Integrovaný obvod & SHT3, SHT31 & \\ \hline   Napájení & Min 2.4V - max. 5.5V & \\ \hline   Rozměry & 9.65 x 40.13 & \\ \hline  }
%obrazek QR kodu
{../img/SHT31V01A_QRcode.png}


\section{Úvodem}
Jedná se o modul založený na IO SHT31V01A, který umožňuje měření relativní vlhkosti a teploty a velkou přesností a stabilitou. Další přesné informace IO je možné vyčíst z oficiálního dokumentačního listu výrobce. Modul obsahuje veškeré potřebné součástky pro správný chod.
\begin{figure}[h!]
\centering
\includegraphics[width=\textwidth]{SHPG.png}
\end{figure}
\begin{figure}[h!]
\centering
\includegraphics[width=\textwidth]{THPG.png}
\end{figure}


\hfill\subsection{Zapojení modulu}
%naimportuje schema
\includepdf[angle=90]{../../hw/sch_pcb/SHT31V01A.pdf}

\section{Osazení a oživení}
\subsection{Osazení}
% totot naimportuje cast, kde jsou osazováky a BOM tabulka
Tady budou předgenerované osazováky a BOM tabulka
\subsubsection{Oživení}
Je potřeba provést kontrolu zda není na plošném spoji zkrat a zda je dobře zapájen IO. Jinak není třeba nic oživovat, pouze připojit a napsat program.

Když je nulovým odporem osazena pozice R4 adresa modulu je 0x44, pokud je osazena pozice R3 je adresa 0x45.
\section{Program}
Vzorový program se nachází ve složce SW modulu. Pro spuštění je potřeba mít nainstalovaný  pyMLAB.

\end{document}
